% Created 2018-03-05 Mon 15:05
\documentclass{article}
\usepackage[mathletters]{ucs}
\usepackage[utf8x]{inputenc}
\usepackage[T1]{fontenc}
\usepackage{fixltx2e}
\usepackage{graphicx}
\usepackage{longtable}
\usepackage{float}
\usepackage{wrapfig}
\usepackage{rotating}
\usepackage[normalem]{ulem}
\usepackage{amsmath}
\usepackage{textcomp}
\usepackage{marvosym}
\usepackage{wasysym}
\usepackage{amssymb}
\usepackage{hyperref}
\tolerance=1000
\date{\today}
\title{Notebook1}
\hypersetup{
  pdfkeywords={},
  pdfsubject={},
  pdfcreator={Emacs 25.3.1 (Org mode 8.2.10)}}
\begin{document}

\maketitle
\tableofcontents

\section{Mathematical Description of the Discrete Cosine Transform}
\label{sec-1}
\begin{enumerate}
\item Brief Overview
\label{sec-1-1}
\begin{itemize}
\item The Discrete cosine transform can represent an image as a sum of
sinusoids with frequencies and magnitudes that differ.
\item The Cosine transform has the property that most of the important
bits of information with an image (or even an audio wave) is
concentrated (Notebook 2 will expand on this idea).
\item Due to this property, the DCT (or the Modified discrete cosine transofrm)
is used to MP3 compression and for JPG compression.
\end{itemize}
\item The Equation
\label{sec-1-2}
\begin{itemize}
\item The Two dimensional DCT of an M-by-N matrix can be expressed as
follows. \\
  C$_{\text{pq}}$ = α$_{\text{p}}$α$_{\text{q}}$ $\sum^{M-n}_{m = 0}\sum^{N-1}_{n = 0} A_{mn}\cos\frac{π(2m + 1)p}{2M}\cos\frac{π(2n + 1)q}{2N}$
\\ $\quad{}$ where
\\ $\quad{} \quad{}$ 0 ≤ p ≤ M - 1
\\ $\quad{} \quad{}$ 0 ≤ q ≤ N - 1
\\ $\quad{} \quad{}$ α$_{\text{p}}$ = $\begin{cases} 1/\sqrt{M} & p = 0 \\
                                            \sqrt{2/M} & 1 ≤ p ≤ M-1
                               \end{cases}$
\\ $\quad{} \quad{}$ α$_{\text{q}}$ = $\begin{cases} 1/\sqrt{N} & p = 0 \\
                                            \sqrt{2/N} & 1 ≤ q ≤ N-1
                               \end{cases}$
\item We can note that if we assign p = n and q = n, they both fit the domain of p and q, and I will be using this for the program
\end{itemize}
\end{enumerate}
% Emacs 25.3.1 (Org mode 8.2.10)
\end{document}