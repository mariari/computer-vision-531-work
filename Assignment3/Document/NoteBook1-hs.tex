% Created 2018-03-30 Fri 19:33
\documentclass{article}
\usepackage[mathletters]{ucs}
\usepackage[utf8x]{inputenc}
\usepackage[T1]{fontenc}
\usepackage{fixltx2e}
\usepackage{graphicx}
\usepackage{longtable}
\usepackage{float}
\usepackage{wrapfig}
\usepackage{rotating}
\usepackage[normalem]{ulem}
\usepackage{amsmath}
\usepackage{textcomp}
\usepackage{marvosym}
\usepackage{wasysym}
\usepackage{amssymb}
\usepackage{hyperref}
\tolerance=1000
\date{\today}
\title{NoteBook1-hs}
\hypersetup{
  pdfkeywords={},
  pdfsubject={},
  pdfcreator={Emacs 25.3.1 (Org mode 8.2.10)}}
\begin{document}

\maketitle
\tableofcontents

\section{Information about Grenade Library}
\label{sec-1}
\begin{itemize}
\item Grenade is a neural network written in pure Haskell for the purpose
of writing fast neural networks that are concise and precise
\item This library gains a lot of expressability and power from its use of
\texttt{Dependent Types}.
\item A Dependent type system allows the a type signature to be dependent
on a value. A language like \href{https://www.idris-lang.org/}{Idris} is built upon this idea.
\begin{itemize}
\item An example of a Dependent type would be an array in which it is a
type error to even try to access an out of range element or a
Tuple where the 2nd element is always greater than the first
element.
\end{itemize}
\item An example of a simple network written in Haskell is
\begin{verbatim}
type SampleNet = Network '[ FullyConnected 10 1, Logit ]
                         '[ 'D1 10, 'D1 1, 'D1 1 ]

randomMyNet :: MonadRandom SampleNet
randomMyNet = randomNetwork
\end{verbatim}
\begin{itemize}
\item Here we make a Network that is fully connected that takes 10
inputs and returns 1 output
\item Notice that fullyConnected takes 10 and 1, which correspond to the
numbers in the 2nd list 'D10 and 'D1. This second list are the
shape. Where 'D1 10 represents a 1D 10 element vector and 'D1 1
represents a 1D vector with 1 element
\item Notice we also have a logit layer which just performs a signmoid
function. Also note that this type takes no term level
information because it doesn't effect the shape of the network at
said point.
\item So really we just made a simple network that does logisitc
regression.
\item The randomMyNet is just a way to initalize with random weights.
\item Notice due to the power of the type system, there is almost no
term level code that needs to be done to construct such a network
\end{itemize}

\item More examples of such a network can be seen on Grenade's \href{https://github.com/HuwCampbell/grenade}{Github}
which shows a few examples most notably a MNIST netowrk with \textasciitilde{}1.5\%
error and a Shakespeare \texttt{RecurrentNetwork}
\end{itemize}

\section{Problems!?!?}
\label{sec-2}
\begin{itemize}
\item So Ι ended up doing small tweaks on \href{https://github.com/HuwCampbell/grenade/blob/master/examples/main/mnist.hs}{the MNIST GitHub Example},
However I wasn't able to properly load the MNIST data.
\begin{verbatim}
λ> x = readMNIST "./train-images-idx3-ubyte"
λ> runExceptT x
*** Exception: ./train-images-idx3-ubyte: hGetContents: invalid argument (invalid byte sequence)
\end{verbatim}
\item I ended up running out of time before figuring out how to properly
format the MNIST data so Ι can read it. Which is a shame because the
example code provides \texttt{readMNIST} and \texttt{parseMINST} along with a test I
could just run. If I got that working I could have just swapped what
the MNIST type was and tested many networks and see how types played
with each other
\item Due to this setback, I ended up just tweaking the example code
slightly and staying with that
\end{itemize}
% Emacs 25.3.1 (Org mode 8.2.10)
\end{document}