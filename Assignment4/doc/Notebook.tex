% Created 2018-04-13 Fri 12:22
\documentclass{article}
\usepackage[mathletters]{ucs}
\usepackage[utf8x]{inputenc}
\usepackage[T1]{fontenc}
\usepackage{fixltx2e}
\usepackage{graphicx}
\usepackage{longtable}
\usepackage{float}
\usepackage{wrapfig}
\usepackage{rotating}
\usepackage[normalem]{ulem}
\usepackage{amsmath}
\usepackage{textcomp}
\usepackage{marvosym}
\usepackage{wasysym}
\usepackage{amssymb}
\usepackage{hyperref}
\tolerance=1000
\usepackage[margin=1.0in]{geometry}
\date{\today}
\title{Notebook}
\hypersetup{
  pdfkeywords={},
  pdfsubject={},
  pdfcreator={Emacs 25.3.1 (Org mode 8.2.10)}}
\begin{document}

\maketitle
\tableofcontents

\section{Defining the Flow field}
\label{sec-1}
\begin{enumerate}
\item With a convolution
\label{sec-1-1}
\begin{itemize}
\item So the plan with this one is to have a n by n window that goes
around the first image, while for every n by n patch, we look at m
by m area of the second area of the second image to find the best
match (an area that minimizes the mean squared difference).
\item In order to do this I wanted to get the degrees of the movement in
terms of a full 360 degrees.
\begin{verbatim}
degrees :: (Ord a, Floating a, Eq a) ⇒ a → a → a
degrees 0    0   = 0
degrees rise 0
  | rise > 0     = 90
  | otherwise    = 180
degrees rise run
  | rise ≥ 0 ∧ run > 0 = calc
  | rise > 0 ∧ run < 0 = calc + 90
  | rise ≤ 0 ∧ run < 0 = calc + 180
  | rise < 0 ∧ run > 0 = calc + 270
  where calc = abs $ atan (rise / run) * 180 / pi
\end{verbatim}
\begin{itemize}
\item here we only really have a few cases which are all self
explanatory.
\end{itemize}
\item Now that we got that out of the way, we shall now talk about the
main function
\begin{verbatim}
convfn n m img1 img2 = R.fromFunction newSize f
  where
    sideSize        = n `div` 2
    edgeSize        = m * n + sideSize
    Z :. i :. j     = extent img1
    newSize         = ix2 (i `div` edgeSize) (j `div` edgeSize) -- this gives the boundary so we stay inside the image
    f (Z :. x :. y) = comp
      where
        centerX     = edgeSize + x * n
        centerY     = edgeSize + y * n
        fromMid ι κ = ix2 (centerX + ι) (centerY + κ)
        extractImg  = extract (fromMid 0 0) (ix2 n n)
        current     = R.computeUnboxedS $ extractImg img1
        sameSpotOn2 = R.computeUnboxedS $ extractImg img2
        comp | current == sameSpotOn2 = 0
             | otherwise              = uncurry degrees added
        -- if the image moved at all then we have to add everything to a priority queue
        added = (fromIntegral lowestI, fromIntegral lowestJ)
          where
            (lowestI, lowestJ) = peek $ foldr insertPQ empty allspots
            allspots           = (,) <$> [negate n*m .. n*m] <*> [negate n*m .. n*m] -- get all points
            insertPQ (ι,κ)     = add diff (ι,κ)
              where
                diff = meanDiff current (extract (fromMid ι κ) (ix2 n n) img2)
\end{verbatim}
\begin{itemize}
\item so we are going to take this function line by line in order to
understand how it works
\item \uline{\texttt{R.fromFunction newSize f}}
\begin{itemize}
\item This line is creating an array with size \texttt{newSize} with default
values defined by a function f that takes coordinates and
constructs the point
\begin{itemize}
\item Note that every point is therefore independent and thus can be
computed in parallel
\end{itemize}
\end{itemize}
\item \uline{sideSize = n `div` 2}
\begin{itemize}
\item here edgeSize is how much an edge goes to either size from the
middle, so if our n is 3, then its side size is 1
\end{itemize}
\item \uline{edgeSize = m * n + sideSize}
\begin{itemize}
\item this part is a little trick, so I don't want the m by m window
to go off the edge of the image, so instead of doing bounds
checking for edge points, I just ignore the size of n m times and
the raidus of n.
\end{itemize}
\item i and j here are just the \texttt{extent} (size) of the first image, n and m
are already taken, so i and j is the next best bet!
\item \uline{newSize = ix2 (i `div` edgeSize) (j `div` edgeSize)}
\begin{itemize}
\item with all these constants defined, we can now define the size of
the the output array. Here we just divide i and j by the edgeSize
and make a new DIM2 shape
\end{itemize}
\item \uline{f (Z :. x :. y) = comp}
\begin{itemize}
\item This is the function that will populate the array, since we now
have the x and y coordinates, we can start to define what this
function does
\item \uline{centerX  centerY}
\begin{itemize}
\item these constants just compute where in the image we are
\end{itemize}
\item \uline{fromMid ι κ = ix2 (centerX + ι) (centerY + κ)}
\begin{itemize}
\item this is just an abstraction that adds a distance from the
middle and generates a shape
\end{itemize}
\item \uline{extractImg = extract (fromMid 0 0) (ix2 n n)}
\begin{itemize}
\item this is yet again another image that takes an array and takes a
n by n patch from the middle
\end{itemize}
\item \uline{current = R.computeUnboxedS \$ extractImg img1}
\begin{itemize}
\item Now we finally have the n by n patch from the first image that
we wish to test against
\end{itemize}
\item So instead of doing a lot of extra work, we define \texttt{sameSpotOn2}
      which grabs the same location in image2, as we can see that in
\item \uline{comp \ldots{} = \ldots{}}
\begin{itemize}
\item that if the current patch is the same as the same patch in
image 2, then we just give back 0, otherwise we do \texttt{uncurry degrees added}
which just calls degrees on added
\end{itemize}
\item \uline{added = (fromIntegral lowestI, fromIntegral lowestJ)}
\begin{itemize}
\item the end result of added is just the min over the m by m window
but to see why, we must see the functions inside of added
\end{itemize}
\item \uline{allspots = (,) <\$> [negate n*m .. n*m] <*> [negate n*m .. n*m]}
\begin{itemize}
\item Here we are doing a little fun trick, where we generate the
range -n*m to n*m and then using map (<\$>) to make the entire
range a 1 argument function
\begin{itemize}
\item (,) <\$> [-2..2] : (Enum a, Num a) ⇒ [b → (a, b)]
\end{itemize}
\item and then we use the applicative (one can think of the
applicative \texttt{<*>} as the cross product that does any arbitrary
functions instead of just ,) we get every combination of -n*m
to n*m
\begin{itemize}
\item (,) <\$> [-1..1] <*> [-1..1] = [(-1,-1),(-1,0),(-1,1),(0,-1),(0,0),(0,1),(1,-1),(1,0),(1,1)]
\end{itemize}
\end{itemize}
\item \uline{insertPQ (ι,κ) = add diff (ι,κ)}
\begin{itemize}
\item \texttt{diff = meanDiff current (extract (fromMid ι κ) (ix2 n n) img2)}
\item I'm going to fold on the above range, but to do so, we must
first make a function that takes a single element of the range
and adds it to a priority queue. and to do this we just take the
meanDiff (defined after this code block section) between the
current patch and the new patch around the points ι and κ
\item After we get this diff we add the diff as the key with the
value pair (ι,κ)
\end{itemize}
\item Now that we got all this work out of the way we can make sense
of
\item \uline{(lowestI, lowestJ) = peek \$ foldr insertPQ empty allspots}
\begin{itemize}
\item This function just folds over allspots (the generalized) cross
product and starts with an initially empty priority queue with
the insertPQ function, now that everything is added to the
priority queue, we can now just peek at the queue and take what
is lowest in value (lowest in the mean squared difference)
\end{itemize}
\end{itemize}
\item With all these functions defined now f is defined and the entire
function just works! and if one is still confused, try re-reading
from top to bottom again, now that you know what each little
function/constant means
\end{itemize}
\item I did a few test cases for this, so I'll include 1 of them
\begin{verbatim}
computeUnboxedS (convfn 3 1 (fromListUnboxed (Z :. (10 :: Int) :. (10 :: Int)) [0..99])
                            (fromListUnboxed (Z :. (10 :: Int) :. (10 :: Int)) [0..99]))
\end{verbatim}
\begin{itemize}
\item and thankfully it gave me back the correct size of the output
\begin{itemize}
\item AUnboxed ((Z :. 2) :. 2) [0.0,0.0,0.0,0.0]
\end{itemize}
\end{itemize}
\item The only thing left to define is the meanDiff I used in convfn
\begin{verbatim}
meanDiff :: (Source r c, Source r2 c, Floating c) ⇒ Array r DIM2 c → Array r2 DIM2 c → c
meanDiff as = √ ∘ (/ fromIntegral (i * j)) ∘ sumAllS ∘ R.zipWith (\x y → abs (x^2 - y^2)) as
  where Z :. i :. j = R.extent as
\end{verbatim}
\begin{itemize}
\item meanDiff just takes 2 arrays and basically just runs the formula
RMSE(a,b) = $\sqrt{\frac{\sum_{t=0}^{n-1}((a_t - b_t)^{2})}{n}}$
\end{itemize}
\end{itemize}
\item With Gradient constraintp
\label{sec-1-2}
\end{enumerate}
% Emacs 25.3.1 (Org mode 8.2.10)
\end{document}